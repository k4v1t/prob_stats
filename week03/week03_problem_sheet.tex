\documentclass[11pt]{article}
% -------------------------------
% Shared packages and commands
% -------------------------------

% --- Page size & margins ---
\usepackage[a4paper,top=2cm,bottom=2.5cm,left=2cm,right=2cm]{geometry}

% --- Title formatting ---
\usepackage{titling}

% Move the whole title block up
\setlength{\droptitle}{-2em}

% Keep author and date on their own lines, just reduce the gaps
\pretitle{\begin{center}\large\bfseries}%
\posttitle{\par\end{center}\vspace{-1em}} % space after title
\preauthor{\begin{center}\large}%
\postauthor{\par\end{center}\vspace{-1em}} % space after author
\predate{\begin{center}\small}%
\postdate{\par\end{center}\vspace{-1em}} % space after date block

% --- Code listings setup ---
\usepackage{listings}
\usepackage{xcolor}

\lstset{
  language=Python,
  basicstyle=\ttfamily\footnotesize,
  keywordstyle=\color{blue},
  commentstyle=\color{gray},
  stringstyle=\color{orange},
  breaklines=true,
  frame=single,
  columns=fullflexible
}

\usepackage{titlesec}
\titlespacing*{\section}{0pt}{1.2ex plus .2ex minus .2ex}{0.8ex plus .1ex}
\titlespacing*{\subsection}{0pt}{1ex plus .2ex minus .2ex}{0.5ex plus .1ex}


% --- Math & theorem environments ---
\usepackage{amsmath, amssymb, amsthm}

% --- Graphics & tables ---
\usepackage{graphicx}
\usepackage{booktabs}

% --- Hyperlinks ---
\usepackage{hyperref}
\hypersetup{
  colorlinks=true,
  linkcolor=blue,
  urlcolor=blue
}

% --- Optional: underlined hyperlinks (remove if not desired) ---
% \usepackage[normalem]{ulem}
% \renewcommand\UrlFont{\uline}

% --- Custom macros (add your own below) ---
% \newcommand{\R}{\mathbb{R}}

\title{MTH794P Probability and Statistics for Data Analytics — Problem Sheet 3}
\author{Kavit Tolia} 
\date{October 16, 2025}

\usepackage{fancyhdr}
\usepackage{lastpage}
\usepackage{enumitem}
\setlist[enumerate,1]{label=(\alph*)}

% ---------- Problem & Solution environments ----------
\newcounter{problem}
\newenvironment{problem}[1][]{%
  \refstepcounter{problem}%
  \bigskip\noindent\textbf{Problem \theproblem. #1}\par\smallskip\noindent
}{\bigskip}

\newenvironment{solution}{%
  \noindent\textit{Solution.}\quad
}{\par\bigskip}

\begin{document}
\maketitle

\pagestyle{fancy}
\fancyhf{}%
\fancyfoot[C]{\thepage\ of \pageref{LastPage}}

% Make the 'plain' style (used on first page, after \maketitle, etc.) match:
\fancypagestyle{plain}{%
  \fancyhf{}%
  \fancyfoot[C]{\thepage\ of \pageref{LastPage}}
  \renewcommand{\headrulewidth}{0pt}
  \renewcommand{\footrulewidth}{0pt}
}

% ---------- Example problems ----------
\begin{problem}
I have three coins in my pocket. Two of them are ordinary fair coins, the third 
has Heads on both sides.
\begin{enumerate}
      \item I take a random coin from my pocket and toss it. What is the probability 
            that it comes up Heads?
      \item Given that it did come up Heads, what is the conditional probability 
            that it is the double-headed coin?
      \item Given that it did come up Heads, what is the conditional probability 
            that a second toss of the same coin also comes up Heads?
      \item Given that two tosses of the same coin both come up Heads, what is 
            the conditional probability that it is the double-headed coin? [Before 
            doing this think do you expect the answer to be larger or smaller than 
            the answer to part (b)? Why?]
      \item Suppose that 100 tosses of the same coin show a Head every time. 
            Without doing any more calculations, say roughly what you expect the 
            conditional probability that it is the double-headed coin to be? How 
            would you explain this in non-mathematical terms?
\end{enumerate}
\end{problem}

\begin{solution} Let us define the events first. Let $D$ be the event that the coin is 
      double-headed and $H_i$ be the event that the $i$-th toss is Heads.
\begin{enumerate}
      \item We can work out this probability as follows:
            \[
            \mathbb{P}(H_1) = \mathbb{P}(H_1 \mid D^{c}) \, \mathbb{P}(D^{c}) + 
                              \mathbb{P}(H_1 \mid D) \, \mathbb{P}(D)
                          = \left(\frac{1}{2}\right) \left(\frac{2}{3}\right)
                              + \left(1\right) \left(\frac{1}{3}\right)
                          = \frac{2}{3}
            \]
      \item Now we want to work out $\mathbb{P}(D \mid H_1)$:
            \[
            \mathbb{P}(D \mid H_1) = \frac{\mathbb{P}(H_1 \mid D) \, \mathbb{P}(D)}{\mathbb{P}(H)}
                        = \frac{\left(1\right) \left(\frac{1}{3}\right)}
                              {\frac{2}{3}}
                        = \frac{1}{2}
            \]
      \item We can work that out as follows: 
            \[
            \mathbb{P}(H_2 \mid H_1) = \frac{\mathbb{P}(H_2 \cap H_1)}
                        {\mathbb{P}(H_1)} = 
                        \frac{\mathbb{P}(H_2 \cap H_1 \mid D^{c}) \, 
                        \mathbb{P}(D^{c}) + \mathbb{P}(H_2 \cap H_1 \mid D) \, 
                        \mathbb{P}(D)}{\mathbb{P}(H_1)}
                        = \frac{(1)(\frac{1}{3}) + (\frac{1}{4})(\frac{2}{3})}{\frac{2}{3}}
                        = \frac{3}{4}
            \]
      \item I expect this probability to be higher than (b), as I'm more certain 
            of the fact that it is now a baised coin.
            \[
            \mathbb{P}(D \mid H_2 \cap H_1) = \frac{\mathbb{P}(H_2 \cap H_1 \mid D) \, \mathbb{P}(D)}{\mathbb{P}(H_2 \cap H_1)}
                        = \frac{\left(1\right) \left(\frac{1}{3}\right)}
                              {\frac{1}{2}} = \frac{2}{3}
            \]
      \item I would expect this conditional probability to be close to 1, as it 
            is highly unlikely that the coin is a fair coin. If it was a fair coin, 
            the chances of getting 100 Heads in a row is close to impossible. The 
            only plausible explanation is that it is the double-headed coin. 
\end{enumerate}
\end{solution}

\begin{problem}
A certain medical condition affect 1\% of the population. A new AI tool for detecting 
this condition from a scan has been developed. It has a 95\% success rate at 
correctly detecting that a person with the condition has it, and only a 2\% chance 
of incorrectly deciding that a healthy person has the condition. A randomly chosen 
person from the population undertakes this test and the test shows positive.
\begin{enumerate}
      \item What is the probability that they do have the condition?
      \item A politician suggests that this test could be used for a national 
            screening programme. What insight into the possible disadvantages of 
            doing this does the calculation of part (a) provide? How could these 
            disadvantages be mitigated?
\end{enumerate}
\end{problem}

\begin{solution}
\begin{enumerate}
      \item Let's define some outcomes. Let $H$ denote a person being healthy and 
            $P$ denote a positive test result. Then,
            \[
            \mathbb{P}(H^{c} \mid P) = \frac{\mathbb{P}(P \mid H^{c})\mathbb{P}(H^{c})}
                  {\mathbb{P}(P)} = \frac{\left(95\%\right)\left(1\%\right)}
                  {\left(95\%\right)\left(1\%\right) + \left(2\%\right)\left(99\%\right)}
                  = 32.5\%
            \]
      \item There is a major disadvantage here in that you have less than 1 in 3 
            chance of a person actually having the condition if they tested 
            positive. This seems odd given the test's high accuracy. But this 
            happens due to the low rate of the condition in the general population. 
            
            We can mitigate this by either: 
            \begin{itemize}
                  \item We can use the test as a first screen, followed by more robust 
                        tests for those who are positive.
                  \item Only screen high-risk groups, where we know the prevalence to be 
                        higher, significantly increasing the probability that a positive 
                        test is indeed someone with the condition.
            \end{itemize}
\end{enumerate}
\end{solution}

\begin{problem}
Suppose that $A$ and $B$ are events with $\mathbb{P}(A) >$ 0 and $\mathbb{P}(B)>$ 0 
and that $\mathbb{P}(A \mid B) > \mathbb{P}(A)$.
\begin{enumerate}
      \item What can you say about $\mathbb{P}(B \mid A)$?
      \item How do you explain the relationship between events $A$ and $B$ which 
            satisfy this property in non-mathematical language?
\end{enumerate}
\end{problem}

\begin{solution}
\begin{enumerate}
      \item Let us write this out:
            \[
            \mathbb{P}(B \mid A) = \frac{\mathbb{P}(A \mid B) \, \mathbb{P}(B)}
                  {\mathbb{P}(A)} > \frac{\mathbb{P}(A) \, \mathbb{P}(B)}
                  {\mathbb{P}(A)} = \mathbb{P}(B)
            \]
            So, we have $\mathbb{P}(B \mid A) > \mathbb{P}(B)$
      \item This tells us that the occurance of one event makes the other more 
            likely to happen. In laymen's terms, $A$ and $B$ are positively
            linked, as they occur together more often. They reinforce each other.
\end{enumerate}
\end{solution}

\newpage

\begin{problem}
\begin{enumerate}
      \item Suppose that $T$ is a discrete random variable measuring the number 
            of days until some event happens. We say that $T$ has a Geometric($p$) 
            distribution if it takes values in 1, 2, 3, ... and the pmf is 
            $\mathbb{P}(T = k) = p \, (1 - p)^{k-1}$.
            \begin{enumerate}[label=(\roman*)]
                  \item What is $\mathbb{P}(T > k)$?
                  \item Show that for any $a, k \in \mathbb{N}$, we have 
                        $\mathbb{P}(T > a + k \mid T > a) = \mathbb{P}(T > k)$. 
            \end{enumerate}
      \item Suppose that the lifetime of a component is a continuous random variable 
            $V$ which follows an Exponential distribution with parameter $\lambda$.
            \begin{enumerate}[label=(\roman*)]
                  \item What is $\mathbb{P}(V > k)$?
                  \item Show that for any $a, k \in \mathbb{R}$, we have 
                        $\mathbb{P}(V > a + k \mid V > a) = \mathbb{P}(V > k)$. 
            \end{enumerate}
      \item Why do you think the results in parts (a)(ii) and (b)(ii) are 
            sometimes called the \textit{memoryless property} of the Geometric 
            and Exponential distributions?
\end{enumerate}
\end{problem}

\begin{solution}
\begin{enumerate}
      \item \begin{enumerate}[label=(\roman*)]
            \item Given we have the pmf of $T$ defined, we can work out the cdf:
                  \[
                  F_T(k) = \mathbb{P}(T \le k) = \sum_{i=1}^{k}{p \, (1 - p)^{i - 1}}
                        = p \sum_{j=0}^{k-1}{(1 - p)^{j}} = \frac{p \, (1 - (1 - p)^{k})}
                        {1 - (1 - p)} = {1 - (1 - p)^{k}}
                  \]
                  Using this cdf, we can then work out $\mathbb{P}(T > k)$:
                  \[
                  \mathbb{P}(T > k) = 1 - \mathbb{P}(T \le k) = 1 - F_T(k) = 
                        1 - \left({1 - (1 - p)^{k}}\right) = {(1 - p)^{k}}
                  \]
            \item We can show this as follows:
                  \[
                  \mathbb{P}(T > a + k \mid T > a) = \frac{\mathbb{P}(T > a + k
                        \, \cap \, T > a)}{\mathbb{P}(T > a)}
                        = \frac{\mathbb{P}(T > a + k)}{\mathbb{P}(T > a)}
                        = \frac{(1 - p)^{a + k}}{(1 - p)^{a}} = {(1 - p)^k}
                  \]
                  Which is the same as $\mathbb{P}(T > k)$.
            \end{enumerate}
      \item \begin{enumerate}[label=(\roman*)]
            \item The cdf of an Exponential distribution with parameter $\lambda$ 
                  is given by $1 - e^{-\lambda k}$ for $k \ge$ 0 and 0 otherwise.
                  Given this, we then have:
                  \[
                  \mathbb{P}(V > k) = 1 - \mathbb{P}(V \le k) = 1 - F_V(k) = 
                        1 - (1 - e^{-\lambda k}) = e^{-\lambda k} \text{ for } k 
                        \ge 0 \text{ and 1 otherwise}
                  \]
            \item The proof below is for cases when $a \ge 0$ and $k \ge 0$. If 
                  either $a < 0$ or $k < 0$, then the conditional probability is 
                  trivial.
                  \[
                  \mathbb{P}(V > a + k \mid V > a) = \frac{\mathbb{P}(V > a + k
                        \, \cap \, V > a)}{\mathbb{P}(V > a)}
                        = \frac{\mathbb{P}(V > a + k)}{\mathbb{P}(V > a)}
                        = \frac{e^{-\lambda(a + k)}}{e^{-\lambda a}} = 
                        {e^{-\lambda k}}
                  \]
                  Which is the same as $\mathbb{P}(V > k)$.
            \item This is called the \textit{memoryless property} because the 
                  probability of waiting an additional amount of time does not 
                  depend on how much time has already passed. So, the distribution 
                  has no ``memory" of the past and the process ``restarts" at 
                  each point in time.
            \end{enumerate}
\end{enumerate}
\end{solution}

\newpage

\begin{problem}
Read this article from mathematical epidemiologist Adam Kucharski's blog: \\
\texttt{https://kucharski.substack.com/p/small-hallucinations-big-problems}
\begin{enumerate}
      \item Using what you learn in this module, redo this analysis in the case 
            that the probability of the unusual event we are looking for is $p$ 
            (rather than the 1 in 1000 that the article uses).
      \item Do you think the author did a good job at explaining the mathematics 
            to a non-mathematical audience?
      \item Look at the Student Forum on the module QMPlus page and make a comment 
            (which could be about your answer to parts (a) or (b) of this question 
            or something else) on the discussion thread about this article.
\end{enumerate}
\end{problem}

\begin{solution}
\begin{enumerate}
      \item If the probability of the unusual event is $p$, then let us define $E$ 
            as the event occurring and $F$ as it being flagged by the LLM. Then,
            \[
            \mathbb{P}(E \mid F) = \frac{\mathbb{P}(F \mid E) \, \mathbb{P}(E)}
            {\mathbb{P}(F \mid E) \, \mathbb{P}(E) + \mathbb{P}(F \mid E^{c}) \, \mathbb{P}(E^{c})}
            = \frac{0.99 p}{0.99p + 0.01(1-p)}
            \]
            We can use this formula to understand how the rarity of an event impacts 
            the probability of a flag being correct:
            \begin{table}[H]
            \centering
            \begin{tabular}{lccccccccc}
            \toprule
            $p$ (\%) & 0.1 & 1.0 & 10.0 & 25.0 & 50.0 & 75.0 & 90.0 & 99.0 & 99.9 \\
            \midrule
            $\mathbb{P}(E \mid F)$ (\%) & 9.016 & 50.000 & 91.667 & 97.059 & 99.000 & 99.664 & 99.888 & 99.990 & 99.999 \\
            \bottomrule
            \end{tabular}
            \caption{How $p$ impacts the flag being correct.}
            \end{table}
            Two things worth mentioning from this table:
            \begin{itemize}
                  \item Unless $p$ exceeds 1\%, most flags will be 
                        false alarms! And this is with an optimistic hallucination 
                        rate of 1\%.
                  \item We would need a $p$ of greater than 50\% to have a 99\%
                        confidence that a flag was real. 
            \end{itemize}
            One way to get a high confidence with a low $p$ would be to have 
            multiple independent LLMs being used for flagging an event. However, 
            this relies on the assumption that their hallucinations are uncorrelated. 
            This might not be the case if they share similar architecture or 
            training data.
      \item Yes, I think the author did a very good job at translating Bayesian 
            statistics into non-mathematical language. They explained the scenario 
            clearly, provided their reasoning for every step and highlighted the 
            pitfalls of using just accuracy as a measure of model performance. 
      \item Done!
\end{enumerate}
\end{solution}

\end{document}