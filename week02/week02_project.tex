\documentclass[11pt]{article}
% -------------------------------
% Shared packages and commands
% -------------------------------

% --- Page size & margins ---
\usepackage[a4paper,top=2cm,bottom=2.5cm,left=2cm,right=2cm]{geometry}

% --- Title formatting ---
\usepackage{titling}

% Move the whole title block up
\setlength{\droptitle}{-2em}

% Keep author and date on their own lines, just reduce the gaps
\pretitle{\begin{center}\large\bfseries}%
\posttitle{\par\end{center}\vspace{-1em}} % space after title
\preauthor{\begin{center}\large}%
\postauthor{\par\end{center}\vspace{-1em}} % space after author
\predate{\begin{center}\small}%
\postdate{\par\end{center}\vspace{-1em}} % space after date block

% --- Code listings setup ---
\usepackage{listings}
\usepackage{xcolor}

\lstset{
  language=Python,
  basicstyle=\ttfamily\footnotesize,
  keywordstyle=\color{blue},
  commentstyle=\color{gray},
  stringstyle=\color{orange},
  breaklines=true,
  frame=single,
  columns=fullflexible
}

\usepackage{titlesec}
\titlespacing*{\section}{0pt}{1.2ex plus .2ex minus .2ex}{0.8ex plus .1ex}
\titlespacing*{\subsection}{0pt}{1ex plus .2ex minus .2ex}{0.5ex plus .1ex}


% --- Math & theorem environments ---
\usepackage{amsmath, amssymb, amsthm}

% --- Graphics & tables ---
\usepackage{graphicx}
\usepackage{booktabs}

% --- Hyperlinks ---
\usepackage{hyperref}
\hypersetup{
  colorlinks=true,
  linkcolor=blue,
  urlcolor=blue
}

% --- Optional: underlined hyperlinks (remove if not desired) ---
% \usepackage[normalem]{ulem}
% \renewcommand\UrlFont{\uline}

% --- Custom macros (add your own below) ---
% \newcommand{\R}{\mathbb{R}}

\title{MTH794P Probability and Statistics for Data Analytics — Week 02 Project}
\author{Kavit Tolia} 
\date{October 9, 2025}

\usepackage{fancyhdr}
\usepackage{lastpage}
\usepackage{enumitem}
\setlist[enumerate,1]{label=(\alph*)}

% ---------- Problem & Solution environments ----------
\newcounter{problem}
\newenvironment{problem}[1][]{%
  \refstepcounter{problem}%
  \bigskip\noindent\textbf{Problem \theproblem. #1}\par\smallskip\noindent
}{\bigskip}

\newenvironment{solution}{%
  \noindent\textit{Solution.}\quad
}{\par\bigskip}

\begin{document}
\maketitle

\pagestyle{fancy}
\fancyhf{}%
\fancyfoot[C]{\thepage\ of \pageref{LastPage}}

% Make the 'plain' style (used on first page, after \maketitle, etc.) match:
\fancypagestyle{plain}{%
  \fancyhf{}%
  \fancyfoot[C]{\thepage\ of \pageref{LastPage}}
  \renewcommand{\headrulewidth}{0pt}
  \renewcommand{\footrulewidth}{0pt}
}

% ---------- Example problems ----------
\begin{problem}
A continuous random variable X has pdf of the form:
      \[
      f_{X}(x) = 
      \begin{cases}
      0 & \text{if } x < \frac{1}{2}, \\
      \frac{C}{x^2}  & \text{if } x \ge \frac{1}{2}
      \end{cases}
      \]
      for some constant \textit{C}.
\begin{enumerate}
      \item What is the value of \textit{C}?
      \item What is $f_{X}(\frac{1}{2})$?
      \item Find the cdf of X.
      \item Write down a couple of ways you could check your answer to (c)
            is plausible. Perform those checks and revisit your answer to (c)
            if necessary.
      \item How would you calculate $\mathbb{P}(1<X<3)$ using the cdf?
      \item How would you calculate $\mathbb{P}(1<X<3)$ using the pdf?
\end{enumerate}
\end{problem}

\begin{solution}
\begin{enumerate}
      \item For $f_{X}(x)$ to be a pdf, we need:
            \[
            \int_{-\infty}^{\infty} f_{X}(x)\,dx = 1
            \Rightarrow \int_{\frac{1}{2}}^{\infty} \frac{C}{x^2}\,dx = 1
            \Rightarrow \left[-\frac{C}{x}\right]_{x=\tfrac{1}{2}}^{\infty} = 1
            \Rightarrow C = \frac{1}{2}
            \]
      \item No, outcomes with the different number of tosses are not equally likely. 
            For example, \textit{HH} occurs with probability 0.25 whereas \textit{TTT} 
            occurs with probability 0.125.
      \item The event \textit{E} ``You toss the coin exactly four times'' can be written as:
            \[
            E = \{\;TTHH,\;TTHT,\;THTH,\;THTT,\;HTTH,\;HTTT\,\}.
            \]
      \item We can define the random variable X as ``the number of tosses required until the game stops'':
            \[
            X : \Omega \to \{2,3,4\}, \qquad
            X(\omega) = \text{number of coin tosses until the experiment stops}.
            \]
\end{enumerate}
\end{solution}

\begin{problem}[Order of 3 runners in a race]
Three runners, Amy, Bea and Cate, take part in a race. The order in which they finish
is recorded.
\begin{enumerate}
      \item Write down the sample space for this experiment.
      \item Write down the event Amy finishes ahead of Cate as a set.
      \item Write down another event both in words and as a set.
      \item Suppose you had \textit{n} runners rather than 3. What is the sample space now 
            and how many elements does it contain?
\end{enumerate}
\end{problem}

\begin{solution}
For the answers, we will denote Amy as A, Bea as B and Cate as C.
\begin{enumerate}
      \item The sample space $\Omega$ for this space is:
            \[
            \Omega = \{\;ABC,\;ACB,\;BAC,\;BCA,\;CAB,\;CBA\,\}.
            \]
      \item The event \textit{E} that Amy finishes before Cate can be written as:
            \[
            E = \{\;ABC,\;ACB,\;BAC\,\}.
            \]        
      \item Another event X can be described as ``Bea finishes last'':
            \[
            X = \{\,ACB,\;CAB\,\}.
            \]
      \item If there are \textit{n} runners, the sample space will be made up of all
            combinations of the \textit{n} runners, and it will have \textit{n}! elements.
\end{enumerate}
\end{solution}

\begin{problem}[Probability with three events]
Let $A$, $B$, and $C$ be events with
\[
\begin{array}{l}
\mathbb{P}(A)=0.7,\quad \mathbb{P}(B)=0.6,\quad \mathbb{P}(C)=0.5,\quad \mathbb{P}(A\cap B)=0.4,\\
\mathbb{P}(A\cap C)=0.3,\quad \mathbb{P}(B\cap C)=0.3,\quad \mathbb{P}(A\cap B\cap C)=0.2.
\end{array}
\]
Calculate the following:
\begin{enumerate}
      \item $\mathbb{P}(A\cup B)$
      \item $\mathbb{P}(A\setminus B)$
      \item The probability that neither of $A$ and $B$ occur.
      \item $\mathbb{P}(A\cup B\cup C)$
      \item The probability that exactly two of $A$, $B$, and $C$ occur.
\end{enumerate}
\end{problem}

\begin{solution}
\begin{enumerate}
      \item $\mathbb{P}(A\cup B) = \mathbb{P}(A) + \mathbb{P}(B) - \mathbb{P}(A\cap B)
            = 0.7 + 0.6 - 0.4 = 0.9$
      \item $\mathbb{P}(A\setminus B) = \mathbb{P}(A) - \mathbb{P}(A\cap B) 
            = 0.7 - 0.4 = 0.3$
      \item $\mathbb{P}($neither $A$ nor $B$$) = \mathbb{P}\bigl((A\cup B)^{c}\bigr)
            = 1 - \mathbb{P}(A\cup B) = 1 - 0.9 = 0.1$
      \item $\mathbb{P}(A\cup B\cup C) = \mathbb{P}(A) + \mathbb{P}(B) + \mathbb{P}(C)
            - \mathbb{P}(A\cap B) - \mathbb{P}(B\cap C) - \mathbb{P}(A\cap C) + 
            \mathbb{P}(A\cap B\cap C) = 0.7 + 0.6 + 0.5 - 0.4 - 0.3 - 0.3 + 0.2 = 1$
      \item $\mathbb{P}(\text{Exactly } 2) = \mathbb{P}(A\cap B) + \mathbb{P}(B\cap C) + 
            \mathbb{P}(A\cap C) - \text{3 }\mathbb{P}(A\cap B\cap C) = 
            0.4 + 0.3 + 0.3 - 3(0.2) = 0.4$
\end{enumerate}
\end{solution}

\begin{problem}[Two random variables]
You have a choice of picking one of the two processes, each of which produces a random 
number. You would prefer a higher number as the outcome. If you pick the first process 
the outcome is random variable $X$; if you pick the second process the outcome is random 
variable $Y$. The random variables $X$ and $Y$ have pmfs as follows:
\[
\begin{array}{c|cccc}
n      & 0   & 1   & 2   & 3 \\ \hline
\mathbb{P}(X=n) & 0.1 & 0.2 & 0.3 & 0.4
\end{array}
\]
\[
\begin{array}{c|cccc}
n      & 0   & 1   & 2   & 3 \\ \hline
\mathbb{P}(Y=n) & 0.2 & 0.2 & 0.1 & 0.5
\end{array}
\]
\begin{enumerate}
      \item Which procedure would you pick?
      \item Give some reasons to justify your answer.
      \item Suppose that these two processes represent two possible medical procedures 
            to treat a condition, with the numerical outcome being the quality of life 
            of a patient after the treatment. How does this extra context change the 
            way you think about the choice between these procedures?
\end{enumerate}
\end{problem}

\begin{solution}
\begin{enumerate}
      \item I would pick process $X$.
      \item $\mathbb{E}(X) = \sum_n n\, \mathbb{P}(X=n) = 
            0(0.1) + 1(0.2) + 2(0.3) + 3(0.4) = 2$ \\
            $\mathbb{E}(Y) = \sum_n n\, \mathbb{P}(Y=n) = 
            0(0.2) + 1(0.2) + 2(0.1) + 3(0.5) = 1.9$ \\
            $X$ has a higher expected value than $Y$, so I chose X.
      \item Given this context, we also need to understand how the variance behaves 
            for each of these processes. \\
            $\mathbb{E}(X) = 2, \, \mathbb{E}(X^2)=5 \implies \mathbb{V}(X)=1$ \\
            $\mathbb{E}(Y) = 1.9, \, \mathbb{E}(Y^2)=5.1 \implies \mathbb{V}(Y)=1.49$ \\
            We can see that $X$ has a higher expected value and a lower variance, so 
            I wouldn't change my choice of picking $X$ as the appropriate process.
\end{enumerate}
\end{solution}

\begin{problem}[Winnings per tossing coins]
\begin{enumerate}
      \item Consider the following game. You pick an amount of money $n$. Then we toss 
            a fair coin. If it comes up Heads, I give you $\pounds n$; if it comes up 
            Tails, you give me $\pounds n$. We repear the game (you can choose a different 
            amount each time) until you decide to stop. \\
            You decide to adopt the following strategy:
            \begin{itemize}
                  \item On the first go stake $\pounds 1$.
                  \item If you win stop.
                  \item If you lose then double your stake on the next game.
                  \item Repeat this (doubling your stake after each loss) until you win.
            \end{itemize}
            You argue as follows:
            \begin{itemize}
                  \item However many turns the game lasts you will win $\pounds 1$. For 
                        example, if it takes 3 turns before the coin comes up Heads, your 
                        total gain in pounds is $-1-2+4=1$. So you are guaranteed to make $\pounds 1$.
                  \item The game shouldn't last too long. If we let $T$ be the number of 
                        times the coin is tossed then you remember (or look it up) the T 
                        has a Geometric distribution with parameter 1/2 so $\mathbb{E}(T)=2$.
                  \item Since in round $r$ you stake $\pounds 2^{r-1}$, the expectation of 
                        the amount of money you expect to risk in the final round is 
                        $\mathbb{E}(2^{T-1})$ which is also small.
            \end{itemize}
            What do you think of each stage of this argument? Is this a sensible strategy 
            to use?
      \item Here is another coin game. We toss a coin until the first time it comes up heads. 
            Support this is on toss number $N$. If $N$ is even, I pay you $\pounds 2^N$; if $N$ 
            is odd, you pay me $\pounds 2^N$. Let $W$ be your total winnings (which could be negative 
            if you end up paying me!). Find the pmf of the random variable $W$. What can you say 
            about $\mathbb{E}(W)$?
\end{enumerate}
\end{problem}

\begin{solution}
\begin{enumerate}
      \item Let's go through each of the stages of the argument in detail here: 
            \begin{itemize}
                  \item The first stage of this argument works under the assumption of an unlimited 
                        bankroll (no cap on losses), no table limits  and allowing to stop at the 
                        first win. Given these assumptions, you are indeed guaranteed to make $\pounds 1$.
                  \item If $T$ is the number of times the coin is tossed, then $T$ follows a geometric 
                        distribution with parameter 1/2. The expected value of a geometric distribution 
                        with parameter $p$ is $1/p$. So, we have $\mathbb{E}(T)=2$. While you could infer 
                        that the ``game shouldn't last long'', we do have to be mindful of the tail risk 
                        event of a huge $2^k$ loss with a rare probability $2^{-k}$. 
                  \item Let's start with the geometric random variable $T$, which has $p=1/2$. We have: \\
                        $\mathbb{E}(2^{T-1})=\sum_t 2^{t-1} \mathbb{P}(T=t)=\sum_t 2^{t-1}(1/2)^{t}=\sum_t (1/2)= \infty $ 
            \end{itemize}
            So, even though the mean of $T$ is 2, the expectation of the final stake is infinte. This is the 
            quantitative reason why the guaranteed $\pounds 1$ is misleading as it requires an infinite bankroll.
      \item The pmf of $W$ is as follows:
            \[
            \begin{array}{c|ccccccccc}
                  w & -2 & -2^{3} & -2^{5} & -2^{7} & \cdots & 2^{2} & 2^{4} & 2^{6} & \cdots \\ \hline
                  \mathbb{P}(W=w) & \tfrac12 & \tfrac{1}{2^{3}} & \tfrac{1}{2^{5}} & \tfrac{1}{2^{7}} & \cdots
                  & \tfrac{1}{2^{2}} & \tfrac{1}{2^{4}} & \tfrac{1}{2^{6}} & \cdots
            \end{array}
            \]
            This means the expectation can be written as: \\
            $\mathbb{E}(W) = \sum_n ((-1)^{n}2^{n})(1/2)^{n} = \sum_n (-1)^{n}$ \\
            This is also known as the Grandi's series and it does not converge. So $\mathbb{E}(W)$ is undefined. \\
            An undefined expectation seems odd, especially given the game seems fair and often ends quickly. 
            However, the possibility of a huge payout or loss (because the stake doubles forever in theory) 
            makes the idea of an expected outcome meaningless. You cannot define a sensible long-run average 
            profit or loss. 
\end{enumerate}
\end{solution}

\end{document}