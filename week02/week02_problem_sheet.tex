\documentclass[11pt]{article}
% -------------------------------
% Shared packages and commands
% -------------------------------

% --- Page size & margins ---
\usepackage[a4paper,top=2cm,bottom=2.5cm,left=2cm,right=2cm]{geometry}

% --- Title formatting ---
\usepackage{titling}

% Move the whole title block up
\setlength{\droptitle}{-2em}

% Keep author and date on their own lines, just reduce the gaps
\pretitle{\begin{center}\large\bfseries}%
\posttitle{\par\end{center}\vspace{-1em}} % space after title
\preauthor{\begin{center}\large}%
\postauthor{\par\end{center}\vspace{-2em}} % space after author
\predate{\begin{center}\small}%
\postdate{\par\end{center}\vspace{-1em}} % space after date block

% --- Code listings setup ---
\usepackage{listings}
\usepackage{xcolor}

\lstset{
  language=Python,
  basicstyle=\ttfamily\footnotesize,
  keywordstyle=\color{blue},
  commentstyle=\color{gray},
  stringstyle=\color{orange},
  breaklines=true,
  frame=single,
  columns=fullflexible
}

\usepackage{titlesec}
\titlespacing*{\section}{0pt}{1.2ex plus .2ex minus .2ex}{0.8ex plus .1ex}
\titlespacing*{\subsection}{0pt}{1ex plus .2ex minus .2ex}{0.5ex plus .1ex}


% --- Math & theorem environments ---
\usepackage{amsmath, amssymb, amsthm}

% --- Graphics & tables ---
\usepackage{graphicx}
\usepackage{booktabs}
\usepackage{float}

% --- Hyperlinks ---
\usepackage{hyperref}
\hypersetup{
  colorlinks=true,
  linkcolor=blue,
  urlcolor=blue
}

% --- Optional: underlined hyperlinks (remove if not desired) ---
% \usepackage[normalem]{ulem}
% \renewcommand\UrlFont{\uline}

% --- Custom macros (add your own below) ---
% \newcommand{\R}{\mathbb{R}}

\title{MTH794P Probability and Statistics for Data Analytics — Problem Sheet 2}
\author{Kavit Tolia} 
\date{October 9, 2025}

\usepackage{fancyhdr}
\usepackage{lastpage}
\usepackage{enumitem}
\setlist[enumerate,1]{label=(\alph*)}

% ---------- Problem & Solution environments ----------
\newcounter{problem}
\newenvironment{problem}[1][]{%
  \refstepcounter{problem}%
  \bigskip\noindent\textbf{Problem \theproblem. #1}\par\smallskip\noindent
}{\bigskip}

\newenvironment{solution}{%
  \noindent\textit{Solution.}\quad
}{\par\bigskip}

\begin{document}
\maketitle

\pagestyle{fancy}
\fancyhf{}%
\fancyfoot[C]{\thepage\ of \pageref{LastPage}}

% Make the 'plain' style (used on first page, after \maketitle, etc.) match:
\fancypagestyle{plain}{%
  \fancyhf{}%
  \fancyfoot[C]{\thepage\ of \pageref{LastPage}}
  \renewcommand{\headrulewidth}{0pt}
  \renewcommand{\footrulewidth}{0pt}
}

% ---------- Example problems ----------
\begin{problem}
A continuous random variable $X$ has pdf of the form:
      \[
      f_{X}(x) = 
      \begin{cases}
      0 & \text{if } x < \frac{1}{2}, \\
      \frac{C}{x^2}  & \text{if } x \ge \frac{1}{2}
      \end{cases}
      \]
      for some constant \textit{C}.
\begin{enumerate}
      \item What is the value of \textit{C}?
      \item What is $f_{X}(\frac{1}{2})$?
      \item Find the cdf of X.
      \item Write down a couple of ways you could check your answer to (c)
            is plausible. Perform those checks and revisit your answer to (c)
            if necessary.
      \item How would you calculate $\mathbb{P}(1<X<3)$ using the cdf?
      \item How would you calculate $\mathbb{P}(1<X<3)$ using the pdf?
\end{enumerate}
\end{problem}

\begin{solution}
\begin{enumerate}
      \item For $f_{X}(x)$ to be a pdf, we need:
            \[
            \int_{-\infty}^{\infty} f_{X}(x)\,dx = 1
            \Rightarrow \int_{\frac{1}{2}}^{\infty} \frac{C}{x^2}\,dx = 1
            \Rightarrow \left[-\frac{C}{x}\right]_{x=\tfrac{1}{2}}^{\infty} = 1
            \Rightarrow C = \frac{1}{2}
            \]
      \item $f_{X}(\frac{1}{2}) = \frac{\frac{1}{2}}{(\frac{1}{2})^2} = 2$
      \item Let us define the cdf of X as $F_X(x)$, then we have:
            \[
            F_x(x) = \int_{-\infty}^{x} f_{X}(t)\,dt
            = \int_{-\infty}^{x} \frac{1}{2t^2}\,dt
            = \left[-\frac{1}{2t}\right]_{t=\frac{1}{2}}^{x}
            = 1 - \frac{1}{2x}
            \]
            So, we have:
            \[
            F_{X}(x) = 
            \begin{cases}
            0 & \text{if } x < \frac{1}{2}, \\
            1 - \frac{1}{2x}  & \text{if } x \ge \frac{1}{2}
            \end{cases}
            \]            
      \item To check that the cdf is valid, we can do the following checks:
            \begin{itemize}
            \item Non-decreasing: We can see that as $x \ge \frac{1}{2}$ increases, $\frac{1}{2x}$
                  decreases. This means $F_X(x)$ increases as it is negatively linked
                  to $\frac{1}{2x}$. So the function is indeed non-decreasing.
            \item Limits at infinity: It's clear that $\lim_{x \to -\infty}F_X(x)=0$ due to
                  how the function is defined. As for the other side, as $x$ approaches infinity
                  we have $\frac{1}{2x}$ approaching 0, and so $F_X(x)$ approaching 1.
            \end{itemize}
      \item We can calculate $\mathbb{P}(1<X<3)$ using the cdf as follows:
            \[
            \mathbb{P}(1<X<3) = \mathbb{P}(X \le 3) - \mathbb{P}(X \le 1)
            = F_X(3) - F_X(1) = 1 - \frac{1}{2(3)} - 1 + \frac{1}{2(1)} 
            = \frac{1}{3}
            \]
      \item If we wanted to do the same thing using a pdf, we can do it by integrating it:
            \[
            \mathbb{P}(1<X<3) = \mathbb{P}(1 \le X \le 3) 
            = \int_{1}^{3} f_{X}(x)\,dx = \int_{1}^{3} \frac{1}{2x^2}\,dx
            = \left[-\frac{1}{2x}\right]_{x=1}^{3}
            = -\frac{1}{6} + \frac{1}{2} = \frac{1}{3}
            \]
\end{enumerate}
\end{solution}

\begin{problem}
A continuous random variable $X$ has cdf
      \[
      F_{X}(x) = 
      \begin{cases}
      0 & \text{if } x < 0, \\
      \frac{x}{6}  & \text{if } 0 \le x < 3, \\
      \frac{1}{2} & \text{if } 3 \le x < 4, \\
      \frac{x-2}{4} & \text{if } 4 \le x < 6, \\
      1 & \text{if } x \ge 6
      \end{cases}
      \]
\begin{enumerate}
      \item Find the pdf of $X$.
      \item Write down the expressions for $\mathbb{E}(X)$ and Var($X$) and compute them.
      \item Describe the distribution this random variable follows in words in a way 
            that would make sense to a non-mathematician.
\end{enumerate}
\end{problem}

\begin{solution}
\begin{enumerate}
      \item The pdf of X, $f_X(x)$ is defined as the derivative of its cdf. So, we have:
            \[
            f_{X}(x) = 
            \begin{cases}
            \frac{1}{6}  & \text{if } 0 \le x < 3, \\
            \frac{1}{4} & \text{if } 4 \le x < 6, \\
            0 & \text{if } x<0, \, 3 \le x < 4, \, \text{or} \, x \ge 6
            \end{cases}
            \]
      \item The expectation of $X$ is defined as follows:
            \[
            \mathbb{E}(X) = \int_{-\infty}^{\infty}xf_X(x)\,dx
            = \int_{0}^{3} \frac{x}{6}\,dx + \int_{4}^{6} \frac{x}{4}\,dx
            = \left[\frac{x^2}{12}\right]_{x=0}^{3} + 
                  \left[\frac{x^2}{8}\right]_{x=4}^{6}
            = \frac{9}{12} + \frac{36}{8} - \frac{16}{8} = 3.25
            \]
            The variance of $X$ is defined as Var($X$) = $\mathbb{E}(X^2) - 
            \mathbb{E}(X)^2$. We already have $\mathbb{E}(X)$, so we just need 
            to work out $\mathbb{E}(X)^2$:
            \[
            \mathbb{E}(X^2) = \int_{-\infty}^{\infty}{x^2}f_X(x)\,dx
            = \int_{0}^{3} \frac{x^2}{6}\,dx + \int_{4}^{6} \frac{x^2}{4}\,dx
            = \left[\frac{x^3}{18}\right]_{x=0}^{3} + 
                  \left[\frac{x^3}{12}\right]_{x=4}^{6}
            = \frac{27}{18} + \frac{216}{12} - \frac{64}{12} = 14.17
            \]
            This means we have the following for Var($X$):
            \[
            \text{Var($X$)} = \mathbb{E}(X^2) - \mathbb{E}(X)^2
            = 14.17 - (3.25)^2 = 3.60
            \]
      \item Let's assume $X$ is the waiting time (in minutes) for a bus. \\
            The bus never arrives before 0 minutes or after 6 minutes, and there 
            is a gap between 3 and 4 minutes where it never comes too.
            \begin{itemize}
            \item The bus arrives half the time in the interval between 0 to 3 minutes, 
                  with all times in that range being equally likely
            \item The other half of the time, the bus arrives between 4 and 6 minutes,
                  where all times are equally likely too
            \end{itemize}
\end{enumerate}
\end{solution}

\begin{problem}
The number $m \in \mathbb{R}$ is a \textit{median} for the random variable $X$ if
$F_X(m) = \frac{1}{2}$.
\begin{enumerate}
      \item Indicate why every continuous random variable has a median. [This can be
            an informal explanation rather than a rigorous mathematical proof, 
            although if you know some analysis you could try to write a proof.]
      \item Can you say anything about $f_X(m)$ (i.e. the pdf evaluated at $m$)?
\end{enumerate}
\end{problem}

\begin{solution}
\begin{enumerate}
      \item The intuitive explanation here can be as follows: \\
            \textit{
            Given the cdf of $X$ is a function which tells us the probability 
            of $X$ being less than a certain value, there will always be a value
            at which point the probability of being less than that value is exactly
            equal to $\frac{1}{2}$. If this was not the case, then the $X$ is not 
            a continuous random variable.
            } \\
            \\
            In terms of a mathematical proof, $F_X(x)$ is a continuous function 
            in the interval [0, 1] (as it is a cdf). We can then use the \textit{
            Intermediate Value Theorem}, which states that if a function $f$ is 
            continuous in the interval [a, b], then for any value $k$ between 
            $f(a)$ and $f(b)$, there much be at least one value $c$ in the 
            interval (a, b) such that $f(c) = k$. In our case, we have $F_X(x)$ which
            is continuous everywhere, so there will be a number $m \in \mathbb{R}$ 
            such that $F_X(m) = k$, where $k = \frac{1}{2}$.
      \item Given any continuous random variable will have a median, that means it
            is agnostic of the type of distribution followed by $X$. It could be a 
            uniform distribution, normal distribution, or any other type. All this 
            means we cannot really say much about the value of the pdf at $m$.
\end{enumerate}
\end{solution}

\begin{problem}
Let $T$ be the random variable giving the time (in minutes) between consecutive 
customers arriving in a shop. Suppose that $T \sim $ Exp(0.5). Each customer spends
5 minutes in the shop and then leaves. The first customer of the day has just 
entered the shop. 
\begin{enumerate}
      \item What is the probability that the next customer does not arrive until 
            after the first customer has left?
      \item What is the expectation of the time before the next customer arrives?
      \item Is the median of the time before the next customer arrives smaller, 
            larger or the same as the expectation?
      \item What is the probability that the time before the next customer arrives
            is greater than twice its expectation?
      \item How do your answers to (c) and (d) change if the parameter of $T$ changes?
\end{enumerate}
\end{problem}

\begin{solution}
\begin{enumerate}
      \item We are looking for the probability that $T > 5$:
            \[
            \mathbb{P}(T > 5) = 1 - \mathbb{P}(T \le 5)
            = 1 - F_T(5) = 1 - (1 - e^{-\lambda(5)}) = e^{(-2.5)} = 0.08
            \]
      \item $\mathbb{E}(T) = 1 / \lambda = 1 / 0.5 = 2$
      \item Let $m$ be the median of $T$, then we have:
            \[
            F_T(m) = \frac{1}{2}
            \Rightarrow 1 - e^{-\lambda m} = \frac{1}{2}
            \Rightarrow e^{-\lambda m} = \frac{1}{2}
            \Rightarrow -\lambda m = \ln{\frac{1}{2}}
            \Rightarrow m = \frac{\ln{2}}{\lambda} = \frac{\ln{2}}{0.5} = 1.39
            \]
            So, the median is less than the expectation, which makes sense given 
            the shape of the cdf. This result is true for all exponential distributions
            as $\ln{2}$ is 0.69. This means the median of an exponential distribution 
            is always 0.69 times the expectation.
      \item Let's work this out for a generic exponential distribution first:
            \[
            \mathbb{P}(T > 2 \mathbb{E}(T)) = 
            1 - \mathbb{P}(T \le 2 \mathbb{E}(T)) = 
            1 - F_T(2 \mathbb{E}(T)) = 
            1 - (1 - e^{-2\lambda \mathbb{E}(T)}) = 
            e^{-2\lambda (\frac{1}{\lambda})} = \frac{1}{e^2} = 0.14
            \]
            Given this derivation does not depend on $\lambda$, it is always true!
      \item No changes to the answers.
\end{enumerate}
\end{solution}

\begin{problem}
Let $\Omega$ be the right-angled triangle with vertices (0,0), (1,0) and (1,1). 
Let $a$ be a point chosen randomly from within $\Omega$ with the probability that 
$a$ is in any fixed region being proportional to the area of the region. Let $T$ be 
the random variable ``The angle of the line from (0,0) to $a$ makes with the $x$-axis".
Find the cdf and pdf of $T$.
\end{problem}

\begin{solution} \\
Given it is a right-angled triangle with the provided vertices, we know that the 
angle between the hypotenuse and the $x$-axis is $\frac{\pi}{4}$. This means the 
cdf of the random variable $T$ is 0 if $T < 0$ and 1 if $T > \frac{\pi}{4}$.
Now, let's say we want to find the probability that $T \le \theta$, where 
$0 \le \theta \le \frac{\pi}{4}$. This probability, $\mathbb{P}(T \le \theta)$, 
is the same as the probability that $a$ is in the right-angled triangle where the 
angle between the hypotenuse and the $x$-axis is $\theta$, which is given to be 
proportional to the area of the right-angled triangle. The area of this right-angled
triangle is $\frac{1}{2}bh$, where $b=1$ and $h$ can be written as 
$\tan(\theta)$. So, the area of that sub-region becomes $\frac{1}{2}\tan(\theta)$.
\[
\mathbb{P}(T \le \theta) = \frac{\text{Area of sub-region}}{\text{Area of }\Omega}
= \frac{\frac{1}{2}\tan(\theta)}{\frac{1}{2}} = \tan(\theta)
\]
This means the cdf can be defined as:
\[
F_{T}(t) = 
\begin{cases}
0 & \text{if } t < 0, \\
\tan(t) & \text{if } 0 \le t \le \frac{\pi}{4}, \\
1 & \text{if } t > \frac{\pi}{4}
\end{cases}
\]
Given that the pdf is just the derivative of the cdf, we have:
\[
f_{T}(t) = 
\begin{cases}
0 & \text{if } t < 0 \text{ or } t > \frac{\pi}{4}, \\
\sec^{2}(t) & \text{if } 0 \le t \le \frac{\pi}{4}, \\
\end{cases}
\]
\end{solution}

\begin{problem}
Decide whether each of the following statements is true or false. Give a reason 
in either case:
\begin{enumerate}
      \item For any continuous random variable $X$, the values of the pdf are 
            probabilities so $0 \le f_X(x) \le 1$ for all $x \in \mathbb{R}$.
      \item For any continuous random variable, the median is always unique (i.e.
            there is only one $m \in \mathbb{R}$ which satisfies the definition 
            of median).
\end{enumerate}
\end{problem}

\begin{solution}
\begin{enumerate}
      \item False. The pdf of a random variable does not give us probabilities. It 
            provides us the density (can be thought of as probability per unit of $x$), 
            which can take any non-negative value. We can see this in (a) of Problem 1.
      \item False. You can have a constant $F_X(x)=\frac{1}{2}$ between two values, 
            which means every value in that interval is a median. The median is only 
            unique if the cdf $F_X(x)$ is strictly increasing (or the pdf $f_X(x)
            > 0$ everywhere).
\end{enumerate}
\end{solution}

\end{document}